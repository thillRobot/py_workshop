% !TEX TS-program = xelatex
% !TEX encoding = UTF-8 Unicode

% Tennessee Technological University
% ENGR1120-0xx - GSET - Summer 2023
% Tristan Hill - June 04, 2023
% Module 0 - Introduction to Python Programming
% Lecture 0 - Computer Hardware Overview (is there a better name?) 

\documentclass[fleqn]{beamer} % for presentation (has nav buttons at bottom)

\usepackage{/home/thill/courses/py_workshop/modules/py_lectures}

\newcommand{\MNUM}{0\hspace{2mm}} % Module number
\newcommand{\TNUM}{---\hspace{2mm}} % Topic number - single topic for now
\newcommand{\moduletitle}{Introduction to Python} % Titles and Stuff
%\newcommand{\topictitle}{---} 

\newcommand{\sectiontitleI}{Course Structure} % More Titles and Stuff
\newcommand{\sectiontitleII}{Course Topics - Learning Objectives}
\newcommand{\sectiontitleIII}{Anatomy of A Computer}
\newcommand{\sectiontitleIV}{The Python Programming Language}
\newcommand{\sectiontitleV}{Tutorial \MNUM - Hello World}

\setbeamercolor{title in head/foot}{fg=TTUgold} % this needs work...

\title{GSET - Programming with Mr. Hill}
\author{Tristan Hill\vspc \hspc Tennessee Technological University \hspc}
\date{Summer 2023}

\begin{document}

\lstset{language=MATLAB,basicstyle=\ttfamily\small,showstringspaces=false}

\frame{\titlepage \center\begin{framed}\Large \textbf{Module \MNUM - \moduletitle}\end{framed} \vspace{5mm}}

% Section 0 - Outline
\frame{
	
	\large \textbf{Module \MNUM - \moduletitle} \vspace{3mm}\\
	
	\begin{itemize}
	
		\item \hyperlink{sectionI}{\sectiontitleI} \vspc % Section I
		\item \hyperlink{sectionII}{\sectiontitleII} \vspc % Section II
		\item \hyperlink{sectionIII}{\sectiontitleIII} \vspc %Section III
		\item \hyperlink{sectionIV}{\sectiontitleIV} \vspc %Section IV	
		\item \hyperlink{sectionV}{\sectiontitleV} \vspc %Section V
	
	\end{itemize}

}

\section{\sectiontitleI}

	% Section I - Frame I
	\begin{frame}[label=sectionI,containsverbatim] \small
		\frametitle{\sectiontitleI}
		\underline{\large Lectures and Assignments}
		\begin{itemize}
			\item This programming course is organized into topic based modules. Each module contains a lecture with slides, and a tutorial with instructions and example code.  
			
			\item The lectures will be held in class, and after each lecture we will work on the tutorials as a group. After completing each tutorial you will write a short summary (README) documenting your program and the work you have done.
			
			\item In addition to the tutorials, programming challenges will be assigned throughout the course to test the skills you have acquired. After completing each programming assignment you will write a short report documenting your program and the work you have done.  
		\end{itemize}
	\end{frame}

	% Section I - Frame II
	\begin{frame} \small
		\frametitle{\sectiontitleI}
		\underline{\large Exams and Grading}	
		\begin{itemize}
			\item You will take a midterm and a final exam in class, and each exam will have a paper and programming component.
			
			\item Your final grade comes from participation in the tutorials (25\%), completion of the programming challenges(25\%), as well as the two graded exams (25\% + 25\%).     
			
			\item See the published syllabus for the complete course details and policies. 
			
		\end{itemize}	
	\end{frame}

	% Section I - Frame III
	\begin{frame} \small
		\frametitle{\sectiontitleI}
		\underline{\large Collaboration and Group Work}	
		\begin{itemize}
			\item Discussion between peers and the instructors is encouraged for all assignments except during the exams.
			
			\item Some assignments may be completed as a group, but some must be completed as individuals. Ask the instructor if you are unsure. 
			
			\item {\BL Sharing of code between students or is \underline{not allowed} unless the activity is designated as a group assignment by the instructor. \vspace{2mm} \\ The Tennessee Technological University academic misconduct policy clearly states that all submitted work must be original and authored by the student(s) submitting the assignment.} 		
		\end{itemize}	
	\end{frame}



\section{\sectiontitleII}

	% Section II - Frame I
	\begin{frame}[label=sectionII,containsverbatim] \small
		\frametitle{\sectiontitleII}
 		During this course we are going to learn about structured computer programming and the Python Language. The following topics will be covered.
		\begin{itemize}
			
			\item Variables, Expressions, and Assignment
			
			\item Basic Input and Output Operations
			
			\item Computer Memory and Common Data Structures
			
			\item Logic and Selection Statements 
			
			\item Looping and Iteration 
			
			\item Introduction to Object Oriented Programming  
			
		\end{itemize}


	\end{frame}

\section{\sectiontitleIII}

	% Section II - Frame I
	\begin{frame} \small
		\frametitle{\sectiontitleIII}
		
		\underline{Primary Components}\vspace{2.5mm}\\
		\renewcommand{\arraystretch}{1.5}
		\begin{tabular}{|c|c|c|}\hline
			Name\hspace*{15mm}& Function\hspace*{20mm}& Use in Programming \hspace*{15mm}\\ \hline
			&& \\ \hline 
			&& \\ \hline
			&& \\ \hline\hline
			&& \\ \hline
			&& \\ \hline
			&& \\ \hline
			&& \\ \hline
		\end{tabular}
	\end{frame}

\section{\sectiontitleIV}	
	% Section IV - Frame I
	\begin{frame}[label=sectionIV] \small
		\frametitle{\sectiontitleIV}    
		
		\underline{A Few Questions}
		\setbeamertemplate{itemize items}[triangle]
		\begin{itemize}
			
			\item What is a computer program? What is computer programming?
			
			\item Have you ever learned about programming before? If so what languages did you use, and where did you learn about them? What types of problems did you solve?
			
			\item Have you ever learned about Python? Why is Python important for scientists and engineers? 
				\begin{itemize}
					\item 
					\item
					\item
				\end{itemize}			
			
		\end{itemize}
	\end{frame}

\section{\sectiontitleV}	
	% Section V - Frame I
	            \begin{frame}[label=sectionV] \small
		\frametitle{\sectiontitleV}    
	
 \setbeamertemplate{itemize items}[triangle]
                \begin{itemize}

					\item {\bf Overview:} In the first tutorial you are going to choose your programming environment on your computer and write your first program, {\it Hello World}.  		

					\item {\bf Assignment:} Complete the tutorial in the document {\it tutorial\MNUM\_introduction\_py.pdf } on ilearn. You must be able to display a custom message from your program to the terminal or standard output.
                    
                    \item {\bf Deliverable:} Write a brief summary of what you accomplished and what you struggled with the most. Include a screen shot or copy of the output of your program.   
    
                    \item {\bf Next Week:} After completion of Module \MNUM, you are ready to learn about Variables, Expressions and Assignment. \vspc
                    
       
                \end{itemize}
		\end{frame}

\end{document}
