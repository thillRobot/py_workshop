% !TEX TS-program = xelatex
% !TEX encoding = UTF-8 Unicode

% GSET Summer 2023 - Tennessee Technological University
% Tristan Hill - June 04, 2023 - May 30, 2023
% Turtorial 0 - Hello World

\documentclass[12pt]{article}

% Custom Preamble
\usepackage{../../py_tutorials} % .sty is parent folder


% Title and Misc
\newcommand{\MNUM}{0} %Module Number
\newcommand{\MNAME}{Introduction to Python} %Module Name
\newcommand{\TNAME}{Hello World} %Tutorial Name
\newcommand{\SEM}{Summer 2024} %Semester
\pagestyle{myheadings}
\markright{{\large GSET - Programming with Mr. Hill}}

\begin{document}

\thispagestyle{plain}

\begin{center}
   {\bf \large GSET - Programming with Mr. Hill - \SEM} \vspace{5mm}\\
   {\bf \Large \MNAME \hspc -  Tutorial\hspc\MNUM\hspc - \TNAME}\vspace{3mm}\\
   
\end{center}

 \hspace*{3cm}\includegraphics[scale=.6]{charlie_robot_sideview.png} \hspace*{1cm}\includegraphics[scale=.18]{blue_marble.jpg}


\begin{description}[labelindent=1cm]
	
	\item[\textbf{\underline{Overview:}}] \hfill \vspace{3mm}\\
	Traditionally, the first program we write when learning a new language is called {\it Hello World}.
	
	\item[\textbf{\underline{System Requirements:}}] \hfill \vspace{0mm}

\begin{itemize}
	\item {\bf Computer}: A computer is required to complete this tutorial. Any OS should work.
	\item {\bf Internet:} Your computer must be connected to the internet to proceed. It is a good idea to update the system before you begin. 
\end{itemize}

	\item[\textbf{\underline{Disclaimer:}}] \hfill \vspace{0mm}
	
	\begin{itemize}

		\item {\GR\underline{\bf Programming Experience:}} Everyone is different and has a different level of computer experience. This is expected. Do not worry if you feel behind or ahead of your peers.
		 
	\end{itemize}

\item[\textbf{\underline{Part 1 - Programming Environment:}}] \hfill \vspace{0mm}

Choose your programming environment before you get started. You have several options.

Recommended Option:

\begin{itemize}

	\item Work in internet browser - \href{https://www.online-python.com/online_python_compiler}{Online Python Compiler} 

	\item Save code locally at end of each session

\end{itemize}

Advanced Options:
\begin{itemize}

	\item Check if Python is already installed on your system by running the following commands.

	\begin{lstlisting} 
	Python --version 			OR 		Python3 --version
	\end{lstlisting}

	If python is installed the version will be displayed and no further steps are needed. Python3 is required for this course. 

	\item If Python3 is not installed on the system, \href{https://www.python.org/downloads/}{download} and install the appropriate version. 
		
	\item Write programs in text editor, and run programs through terminal.

\end{itemize}	
\newpage

\item[\textbf{\underline{Part 2 - The Python Code:}}] \hfill \vspace{0mm}
\begin{enumerate}
    \item You have been given the following example code. Study the program before running it.
	%\begin{minted}{cpp}
	\begin{lstlisting}
	#GSET - Summer 2023
	#Introduction to Python  
	#Hello World - June 5, 2023

	print('Hello GSET  World')
	\end{lstlisting}
	%\end{minted}

	\item	
Look at each line of the program carefully.
	\begin{itemize}
		\item Lines 1-3: Block Comment 
		\item Line 5: The print function
	\end{itemize}
	
\end{enumerate}

	\item[\textbf{\underline{Part 3 - Testing:}}] \hfill \vspace{0mm}
	\begin{enumerate}
	
		\item Now, run the Python code by pressing the run button.
		
		\item	
		The output of the program is shown in the terminal below.
		\begin{lstlisting} 
		Hello GSET  World

** Process exited - Return Code: 0 **
Press Enter to exit terminal

		\end{lstlisting}
	
		\item Save your code with the download button or use copy and paste. You can view and edit the code in any text editor. Also, save a copy of the program output for your tutorial summary. 

	\end{enumerate}

\item[\textbf{\underline{Tutorial Complete:}}] \hfill \vspace{3mm}\\ 
	Congratulations, after completing {\it Tutorial \MNUM - Hello World}, you have begun learning to program in Python! \\


\newpage
\item[\textbf{\underline{Tutorial Summary:}}] \hfill \vspace{3mm}\\ 
Write a brief summary of what you accomplished and what you struggled with the most. 

Include the following items in the summary:
\begin{itemize}

\item a copy of the output of your program
\item a description of what the program does and how to use it

\end{itemize}


\item[\textbf{\underline{Submission on Teams:}}] \hfill \vspace{3mm}\\ 
Use the appropriate shared folder on Teams to submit your program and summary. Submit the fol1owing items with your TNTech username in the filenames as shown below. \vspace{0mm}\\

\underline{Files for Tutorial 1 (TNTech Username : twhill21)}

\begin{itemize}

\item Tutorial Summary: \textbf{ twhill21\_summary0.txt}

\item Python Source Code: \textbf{ twhill21\_tutorial0.py}

\end{itemize}


\end{description}
\end{document}

