% !TEX TS-program = xelatex
% !TEX encoding = UTF-8 Unicode

% Tennessee Technological University
% ENGR1220-001 - GSET - Summer 2023
% Tristan Hill - June 13, 2023
% Module 6 - Conditionals
% Lecture 6


\documentclass[fleqn]{beamer} % for presentation (has nav buttons at bottom)

%\usepackage{/home/thill/Documents/lectures/cpp_workshop/modules/cpp_lectures}
\usepackage{/mnt/c/Users/thill/Documents/courses/py_workshop/modules/py_lectures}


\newcommand{\MNUM}{6\hspace{2mm}} % Module number
\newcommand{\TNUM}{---\hspace{2mm}} % Topic number - single topic for now
\newcommand{\moduletitle}{Conditionals} % Titles and Stuff
%\newcommand{\topictitle}{---} 

\newcommand{\sectiontitleI}{Logical Statements} % More Titles and Stuff
\newcommand{\sectiontitleII}{Comparison Operations}
\newcommand{\sectiontitleIII}{Branching and Program Flow}
\newcommand{\sectiontitleIV}{Conditional Statements in Python}
\newcommand{\sectiontitleV}{Logical Operators and Compound Statements}

\newcommand{\btVFill}{\vskip0pt plus 1filll}

\setbeamercolor{title in head/foot}{fg=TTUgold} % this needs work...

\title{GSET - Intro to Programming with Python}
\author{Tristan Hill\vspc \hspc Tennessee Technological University \hspc}
\date{Summer 2023}

\begin{document}

\lstset{language=Python,basicstyle=\ttfamily\small,showstringspaces=false}

\frame{\titlepage \center\begin{framed}\Large \textbf{Module \MNUM - \moduletitle}\end{framed} \vspace{5mm}}

% Section 0 - Outline
\begin{frame}
	
	\large \textbf{Module \MNUM - \moduletitle} \vspace{3mm}\\
	
	\begin{itemize}
	
		\item \hyperlink{sectionI}{\sectiontitleI} \vspc % Section I
		\item \hyperlink{sectionII}{\sectiontitleII} \vspc % Section II
		\item \hyperlink{sectionIII}{\sectiontitleIII} \vspc %Section III
		\item \hyperlink{sectionIV}{\sectiontitleIV} \vspc %Section IV	
	 	\item \hyperlink{sectionV}{\sectiontitleV} \vspc %Section V
	
	\end{itemize}

\end{frame}

% Section I
\section{\sectiontitleI}

	% Section I - Frame I
	\begin{frame}[label=sectionI,containsverbatim] \small
		\frametitle{\sectiontitleI}
		What are the four types of sentences?
		
		\begin{enumerate}
			
			\item 
			 
			\item
			
			\item
			
			\item
			
		\end{enumerate}
	
	We are going to discuss one of these in detail.
	
	\end{frame}

	% Section I - Frame II
	\begin{frame}[label=sectionII,containsverbatim] \small
		\frametitle{\sectiontitleI}
		
A {\PR logical statement} is a statement that can be evaluated as {\GR true} or {\RD false}. \vspcc

Why are we discussing this? How are logical statements used in programming? \vspcc

	\end{frame}

	% Section I - Frame II
	\begin{frame} \small
		\frametitle{\sectiontitleI}
		
		A Classic Riddle: Knights and Knaves \vspccc
		
		{\it John and Bill are standing at a fork in the road. John is standing in front of the left road, and Bill is standing in front of the right road. One of them is a knight (truth-teller) and the other a knave (liar), but you don't know which. You also know that one road leads to good, and the other leads to bad. By asking one yes–no question, can you determine the road to good?} 
		\btVFill
		
		\tiny{modified from: \href{https://en.wikipedia.org/wiki/Knights_and_Knaves}{wikipedia} } 	

	\end{frame}	

% Section II
\section{\sectiontitleII}

	\begin{frame}[label=sectionII,containsverbatim] \small
	\frametitle{\sectiontitleII}
	
	If you studied mathematics, then you are familiar with these operators.  \\
			
		\renewcommand*{\arraystretch}{1.5}
		\begin{tabular}{c|c|c} 
			Name&Symbol&Example\\ \hline
			strictly less than&$<$ & \\ \hline
			less than or equal to&$<=$ & \\ \hline
			strictly greater than&$>$ & \\ \hline
			greater than or equal to&$>=$ & \\ \hline
			equal (is equals to)&$==$ & \\ \hline
			not equal (is not equal to)&$!=$ & \\ \hline
			object identity&$is$ & \\ \hline
			negated object identity&$is\hspace{2mm}not$& \\ \hline
		\end{tabular}
		
	
	\btVFill
	
	\tiny{reference: \href{https://docs.python.org/3/library/stdtypes.html}{docs.python.org - Built-in Types} } 	
	
	\end{frame}      


	\begin{frame}[label=sectionII,containsverbatim] \small
	\frametitle{\sectiontitleII}
		\vspace*{5mm}
		Relational operators are used to create logical statements.

		\btVFill
	\end{frame}




% Section III
\section{\sectiontitleIII}	
	% Section III - Frame I
	\begin{frame}[label=sectionIII,containsverbatim] \small
		\frametitle{\sectiontitleIII}    
	
		Previously the programs we have written look like a single line. 
	
		\begin{lstlisting}
		
		...
		
		
		
		
		
		
		
		
		
		
		...			
		
		\end{lstlisting}
		 

		\btVFill
		%\tiny{ref: \href{some link}{some text}} 
	\end{frame}

	% Section III - Frame II                                     
	\begin{frame}[label=sectionIII,containsverbatim] \small
	\frametitle{\sectiontitleIII}    
	
	Commonly Used Flowchart Symbols
	
	\begin{itemize}
		\item Flowline
		\item Terminal
		\item Process
		\item Decision
		\item Input/Output
		\item and many more ... 
		
	\end{itemize}

	{\it Flowcharting is a tool for brainstorming and it can be used for communication and education. A flowchart is not a program. }
	
	
	\btVFill
	\tiny{ref: \href{https://en.wikipedia.org/wiki/Flowchart}{wikipedia}} 
\end{frame}

% Section IV
\section{\sectiontitleIV}	
% Section IV - Frame I
\begin{frame}[label=sectionIV,containsverbatim] \small
\frametitle{\sectiontitleIV}    
quadratic equation example:
\begin{lstlisting}
# get string from user
coefs=input('Type a,b,c and press enter: ')  
a=float(coefs.split(',')[0])  # split items by commas 
b=float(coefs.split(',')[1])  # access with index
c=float(coefs.split(',')[2])  # convert string to float  

x1=(-b+(b**2-4*a*c)**(0.5))/(2*a)  # calculate root values
x2=(-b-(b**2-4*a*c)**(0.5))/(2*a)

print('The first root is:', x1)
print('The second root is:', x2)
\end{lstlisting}

\btVFill
{\tiny \href{https://docs.python.org/3/tutorial/controlflow.html}{docs.python.org - More Flow Control Tools}}

\end{frame}

% Section IV - Frame II
\begin{frame}[label=sectionIV,containsverbatim] \small
\frametitle{\sectiontitleIV}    
quadratic equation example:
\begin{lstlisting}
# determine case from sign of descriminant
if (b**2-4*a*c)>0:
	print('The roots are real numbers')
else:
	print('The roots are complex numbers')
\end{lstlisting}

\btVFill
{\tiny \href{https://docs.python.org/3/tutorial/controlflow.html}{docs.python.org - More Flow Control Tools}}

\end{frame}

% Section V
\section{\sectiontitleV}	
	% Section V - Frame I
	\begin{frame}[label=sectionV,containsverbatim] \small
	\frametitle{\sectiontitleV}    
	
	If you studied programming, then you are probably familiar with these operators.  \vspace{5mm}\\
	
	\renewcommand*{\arraystretch}{1.5}
	\begin{tabular}{c|c|c} 
		Name&Symbol&Example\\ \hline
		conjunction&$\wedge$ & \\ \hline
		disjunction&$\vee$ & \\ \hline
		negation&$\sim$& \\ \hline

	\end{tabular}

	\end{frame}
	

	% Section V - Frame II
	\begin{frame}[label=sectionV,containsverbatim] \small
	\frametitle{\sectiontitleV}    
	
	Compound Logical Statements as conditionals in Python  \vspace{5mm}\\

	\btVFill
	\href{https://docs.python.org/3/reference/compound_stmts.html#}{docs.python.org - Compound Statements}
	\end{frame}


	% Section V - Frame II
	\begin{frame}[label=sectionV,containsverbatim] \small
	\frametitle{\sectiontitleV}    
	
	Compound Statements and Truth Tables.  \vspace{5mm}\\
	
	\renewcommand*{\arraystretch}{1.5}
	\begin{tabular}{|c|c|c|c|c|} \hline
		---&---& Cmpd. Stmt. C& Cmpd. Stmt. D& Cmpd. Stmt. E \\ \hline
		Stmt. A&Stmt. B& & & \\ \hline
		& & & & \\ \hline
		& & & & \\ \hline
		& & & & \\ \hline
		& & & & \\ \hline 
	
	\end{tabular}

	\vspace*{3mm}
	Hint: Use a truth table to solve the Knights and Knaves riddle.

	\end{frame}

\end{document}