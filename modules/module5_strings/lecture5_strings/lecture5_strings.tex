% !TEX TS-program = xelatex
% !TEX encoding = UTF-8 Unicode

% Tennessee Technological University
% ENGR1120-021 - GSET - Summer 2023
% Tristan Hill - June 11, 2023
% Module 5 - Strings
% Lecture 5


\documentclass[fleqn]{beamer} % for presentation (has nav buttons at bottom)

%\usepackage{/home/thill/Documents/lectures/cpp_workshop/modules/cpp_lectures}
\usepackage{/mnt/c/Users/thill/Documents/courses/py_workshop/modules/py_lectures}


\newcommand{\MNUM}{5\hspace{2mm}} % Module number
\newcommand{\TNUM}{---\hspace{2mm}} % Topic number - single topic for now
\newcommand{\moduletitle}{Strings} % Titles and Stuff
%\newcommand{\topictitle}{---} 

\newcommand{\sectiontitleI}{A different type of data} % More Titles and Stuff
\newcommand{\sectiontitleII}{Using Strings in Python }
\newcommand{\sectiontitleIII}{Sequence Operations with Strings}
\newcommand{\sectiontitleIV}{String Methods}
\newcommand{\sectiontitleV}{Example Code}

\newcommand{\btVFill}{\vskip0pt plus 1filll}

\setbeamercolor{title in head/foot}{fg=TTUgold} % this needs work...

\title{GSET - Intro to Programming with Python}
\author{Tristan Hill\vspc \hspc Tennessee Technological University \hspc}
\date{Summer 2023}

\begin{document}

\lstset{language=Python,basicstyle=\ttfamily\small,showstringspaces=false}

\frame{\titlepage \center\begin{framed}\Large \textbf{Module \MNUM - \moduletitle}\end{framed} \vspace{5mm}}

% Section 0 - Outline
\begin{frame}
	
	\large \textbf{Module \MNUM - \moduletitle} \vspace{3mm}\\
	
	\begin{itemize}
	
		\item \hyperlink{sectionI}{\sectiontitleI} \vspc % Section I
		\item \hyperlink{sectionII}{\sectiontitleII} \vspc % Section II
		\item \hyperlink{sectionIII}{\sectiontitleIII} \vspc %Section III
		\item \hyperlink{sectionIV}{\sectiontitleIV} \vspc %Section IV	
	 	\item \hyperlink{sectionV}{\sectiontitleV} \vspc %Section V
	
	\end{itemize}

\end{frame}

% Section I
\section{\sectiontitleI}

	% Section I - Frame I
	\begin{frame}[label=sectionI] \small
		\frametitle{\sectiontitleI}

		We have learned to store and use numerical data in Python. This type of data is common to the scientist and engineer.

		\begin{itemize}
			\item Computations for Analysis
			\item Data Collection and Storage
			\item System Operation and Control
		\end{itemize}

		Computer systems require a different type of data which {\it may} be less familiar to the standard user. The \underline{\hspace{20mm}} and the \underline{\hspace{20mm}} data types have many uses in computer programming.

		\begin{itemize}
			\item
			\item
			\item
		\end{itemize}
		
		\btVFill
		\tiny{reference: \href{}{} } 	
			
	\end{frame}

	% Section I - Frame II
	\begin{frame} \small
		\frametitle{\sectiontitleI}
		
		\href{https://www.asciitable.com/}{ASCII} - American Standard for Information Interchange - convention used for character encoding in most computer systems

		\begin{itemize}
			\item
			\item
		\end{itemize}

		\href{https://home.unicode.org/}{Unicode} - {\it The Unicode Consortium is the standards body for the internationalization of the software and services}   

		\begin{itemize}
			\item
			\item
		\end{itemize}
		
	\end{frame}



% Section II
\section{\sectiontitleII}

	% Section II - Frame I
	\begin{frame}[label=sectionII, containsverbatim] \small
		\frametitle{\sectiontitleII}
		
		{\it Textual data in Python is handled with str objects, or strings. Strings are immutable sequences of Unicode code points. String literals are written in a variety of ways:} - docs.python.org

		\begin{lstlisting}
string1='Single quotes are nice!'

string2="Double quotes allow 'embedded' single quotes" 

string3='''Triple quote (single or double) allow for
multiple line strings'''
		\end{lstlisting}


		\tiny{reference: \href{https://docs.python.org/3/library/stdtypes.html#textseq}{docs.python.org}}
	
	\end{frame}

	% Section II - Frame II
	\begin{frame}[containsverbatim] \small
		\frametitle{\sectiontitleII}

A string is typically made up of multiple letters, numbers, and/or symbols. \\ \vspace{2mm} 
In Python, a string behaves like a \underline{\hspace{15mm}}. This means the items in a string can be accessed using the square brackets [ ] . \\ \vspace{2mm}   		

		\begin{lstlisting}
language='Python 3'

print('The string is:', language) 
print('The first item is:', language[0])
print('The last item is:', language[len(langauge)-1])
		\end{lstlisting}

		\btVFill
		\tiny{ref: \href{some link}{some text}}
	\end{frame}	

		% Section II - Frame III
	\begin{frame}[containsverbatim] \small
		\frametitle{\sectiontitleII}
		
		Redefining Items in a List	

		\begin{lstlisting}
language='Python 3'

language[7]='2'		
		\end{lstlisting}

	\end{frame}

	% Section II - Frame IV
	\begin{frame}[containsverbatim] \small
		\frametitle{\sectiontitleII}

		The previous line will not run. Read the error message carefully.

		\begin{framed}
			\begin{verbatim}
				
				Traceback (most recent call last):
		  File "../lecture5_strings.py", line 23, in <module>
		    language[7]='2'
		TypeError: 'str' object does not support item assignment
				
			\end{verbatim}
		\end{framed}

		\btVFill
		\tiny{ref: \href{}{python.org}}
	\end{frame}	


% Section III
\section{\sectiontitleIII}

	% Section III - Frame I
	\begin{frame}[label=sectionIII, containsverbatim] \small
		\frametitle{\sectiontitleIII}

		Common Sequence Operations

		\begin{itemize}
		\item \lstinline{x in s} - True if an item of s is equal to x, else False 	
		\item \lstinline{s + t} - the concatenation of s and t
		\item \lstinline{len(s)} - length of s 
		\end{itemize}
		
		See the full list in the link below

		\btVFill
		\tiny{ref: \href{https://docs.python.org/3/library/stdtypes.html#common-sequence-operations}{docs.python.org - common sequence operations}}
	\end{frame}

% Section IV
\section{\sectiontitleIV}	
	% Section IV - Frame I
	\begin{frame}[label=sectionIV] \small
		\frametitle{\sectiontitleIV}

		The string object has many useful methods. Here are just a few. \vspace{3mm}

		\begin{itemize}
			\item \lstinline{str.isascii()} - Return True if the string is empty or all characters in the string are ASCII, False otherwise.
			\item \lstinline{str.isdecimal()} - Return True if all characters in the string are decimal characters and there is at least one character, False otherwise.
			\item \lstinline{str.split(sep=None, maxsplit=- 1)} - Return a list of the words in the string, using sep as the delimiter string...
		\end{itemize}
		\vspace*{5mm}

		See the full list by clicking the link below. 

		\btVFill
		\tiny{ref: \href{https://docs.python.org/3/library/stdtypes.html\#string-methods}{docs.python.org - string methods} } 
	\end{frame}

% Section V
\section{\sectiontitleV}	
 	% Section V - Frame I
 	\begin{frame}[label=sectionV,containsverbatim] \small
 	 	\frametitle{\sectiontitleV}    
 	 	
 	 	\vspace*{15mm}
		See {\bf lecture5\_strings.py} for example Python 3 code.	

 	 	\btVFill
 		\tiny{}
 	\end{frame}

\end{document}