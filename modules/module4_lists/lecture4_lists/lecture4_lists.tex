% !TEX TS-program = xelatex
% !TEX encoding = UTF-8 Unicode

% Tennessee Technological University
% ENGR1120-021 - GSET - Summer 2023
% Tristan Hill - June 08, 2023
% Module 4 - Lists
% Lecture 4


\documentclass[fleqn]{beamer} % for presentation (has nav buttons at bottom)

%\usepackage{/home/thill/Documents/lectures/cpp_workshop/modules/cpp_lectures}
\usepackage{/mnt/c/Users/thill/Documents/courses/py_workshop/modules/py_lectures}


\newcommand{\MNUM}{3\hspace{2mm}} % Module number
\newcommand{\TNUM}{---\hspace{2mm}} % Topic number - single topic for now
\newcommand{\moduletitle}{Lists} % Titles and Stuff
%\newcommand{\topictitle}{---} 

\newcommand{\sectiontitleI}{Data Stuctures and Programming} % More Titles and Stuff
\newcommand{\sectiontitleII}{Using Lists in Python}
\newcommand{\sectiontitleIII}{Common Methods}
\newcommand{\sectiontitleIV}{Advanced Methods}
%\newcommand{\sectiontitleV}{Advanced Methods}

\newcommand{\btVFill}{\vskip0pt plus 1filll}

\setbeamercolor{title in head/foot}{fg=TTUgold} % this needs work...

\title{GSET - Intro to Programming with Python}
\author{Tristan Hill\vspc \hspc Tennessee Technological University \hspc}
\date{Summer 2023}

\begin{document}

\lstset{language=MATLAB,basicstyle=\ttfamily\small,showstringspaces=false}

\frame{\titlepage \center\begin{framed}\Large \textbf{Module \MNUM - \moduletitle}\end{framed} \vspace{5mm}}

% Section 0 - Outline
\begin{frame}
	
	\large \textbf{Module \MNUM - \moduletitle} \vspace{3mm}\\
	
	\begin{itemize}
	
		\item \hyperlink{sectionI}{\sectiontitleI} \vspc % Section I
		\item \hyperlink{sectionII}{\sectiontitleII} \vspc % Section II
		\item \hyperlink{sectionIII}{\sectiontitleIII} \vspc %Section III
		\item \hyperlink{sectionIV}{\sectiontitleIV} \vspc %Section IV	
	%	\item \hyperlink{sectionV}{\sectiontitleV} \vspc %Section V
	
	\end{itemize}

\end{frame}

% Section I
\section{\sectiontitleI}

	% Section I - Frame I
	\begin{frame}[label=sectionI] \small
		\frametitle{\sectiontitleI}

		There are many ways to store data in the computer. \vspace{5mm}\\

		Q: We have practiced using single values, but how do we store multiple values? \vspace{5mm}\\

		A: 

		\btVFill
		\tiny{reference: \href{https://docs.python.org/3/tutorial/datastructures.html\#}{docs.python.org} } 	
			
	\end{frame}

	% Section I - Frame II
	\begin{frame} \small
		\frametitle{\sectiontitleI}

		Common Data Stuctures 
		\renewcommand{\arraystretch}{1.5}
		\begin{tabular}{|c|c|} \hline
			Name& Description \hspace{50mm} \\ \hline
			Array&  \\ \hline
			*List&  \\ \hline
			Stack&  \\ \hline
			Queue&  \\ \hline
			*Tuple and Sequence&  \\ \hline
			*Set&  \\ \hline
			*Dictionary&  \\ \hline
		\end{tabular}

		* Python Data Structures
	\end{frame}
		

	% Section I - Frame III
	\begin{frame} \small
		\frametitle{\sectiontitleI}

		Python Data Structures

		\begin{itemize}
			\item List
			\item Tuple
			\item Set
			\item Dictionary
		\end{itemize}

		Note: C++ and MATLAB use arrays, but Python does not.



		\btVFill
		\tiny{reference: \href{https://docs.python.org/3/tutorial/datastructures.html\#}{docs.python.org} } 
	\end{frame}


% Section II
\section{\sectiontitleII}

	% Section II - Frame I
	\begin{frame}[label=sectionII, containsverbatim] \small
		\frametitle{\sectiontitleII}
			
		Initializing a List

		\begin{lstlisting}

buildings = ['Brown', 'Clement', 'Prescott', 'Bruner']

		\end{lstlisting}

	
	\end{frame}

	% Section II - Frame II
	\begin{frame}[containsverbatim] \small
		\frametitle{\sectiontitleII}
		
		Accessing Items in a List	

		\begin{lstlisting}

buildings = ['Brown', 'Clement', 'Prescott', 'Bruner']

print('The north building is', buildings[3])

print('The south building is', buildings[0])

print('The east building is', buildings[1])

print('The west building is', buildings[2])
		\end{lstlisting}

		\btVFill
		\tiny{ref: \href{some link}{some text}}
	\end{frame}	

		% Section II - Frame III
	\begin{frame}[containsverbatim] \small
		\frametitle{\sectiontitleII}
		
		Redefining Items in a List	

		\begin{lstlisting}

buildings[3] = ['New Bruner']

print('The CSC department is in', buildings[3])

		\end{lstlisting}


		\vspace*{10mm}
		Remember the data in the list is {\it mutable}, meaning it can be changed after it has been defined.
	

		\btVFill
		\tiny{ref: \href{some link}{some text}}
	\end{frame}	


% Section III
\section{\sectiontitleIII}

	% Section III - Frame I
	\begin{frame}[label=sectionIII] \small
		\frametitle{\sectiontitleIII}
		

		\btVFill
		\tiny{reference: some ref }

	\end{frame}

	% Section III - Frame II
	\begin{frame} \small
		\frametitle{\sectiontitleIII}
		
		
		
	\end{frame}


% Section IV
\section{\sectiontitleIV}	
	% Section IV - Frame I
	\begin{frame}[label=sectionIV] \small
		 

		\btVFill
		\tiny{reference: more info } 
	\end{frame}

% Section V
%\section{\sectiontitleV}	
%	% Section V - Frame I%
%	\begin{frame}[label=sectionV,containsverbatim] \small
%	\frametitle{\sectiontitleV}    
%	\end{frame}


\end{document}

